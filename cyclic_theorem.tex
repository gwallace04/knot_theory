\documentclass[letterpaper, 11pt]{article}
\usepackage{amsmath}
\usepackage{amssymb}
\usepackage{enumitem}
\usepackage{mathrsfs}
\usepackage{fancyhdr}
\pagestyle{fancy}
\rhead{Gabriel Wallace}
\lhead{Knot Theory}

\newcommand{\N}{\mathbb{N}}
\newcommand{\R}{\mathbb{R}}
\newcommand{\Z}{\mathbb{Z}}
\newcommand{\Q}{\mathbb{Q}}

\newcommand{\cyclic}[1]{\langle #1 \rangle}
\newcommand{\inv}{^{-1}}

\begin{document}
\textbf{Theorem} If $G$ colors a knot and every subgroup of $G$ is normal, then $G$ is cyclic.\\

\emph{Proof.} Let $x$ be a label of a strand. If $G = \langle x \rangle$, then we are done. So assume
$\cyclic{x} \subsetneq G$. Let $y$ label a strand and let $y \in G \setminus \cyclic{x}$. Then $x$
and $y$ are conjugate. Thus, 
$y = gxg\inv$ for some $ g \in G$
and therefore $y \in g \cyclic{x} g\inv$. Since every subgroup of $G$ is normal, then $g\cyclic{x}
g\inv = \cyclic{x}$. A contradiction, showing $y \notin G \setminus \cyclic{x}$, implying that $y
\in \cyclic{x}$.  

Now assume $y \in G$ does not label a strand. Let $L$ be the set of labels of the knot, so
$\cyclic{L} = G$. Since $y$ does not label a strand then $y \in G \setminus L$. But, $L \subseteq
\cyclic{x}$, so $G \subset \cyclic{x}$. A contradiction, showing that $G \cong \cyclic{x}$.  

\end{document}
