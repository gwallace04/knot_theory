\documentclass[letterpaper, 11pt]{article}
% \usepackage[hmargin = 1in, vmargin = 1in]{geometry}
\usepackage{amsmath}
\usepackage{amssymb}
\usepackage{scrextend}
\usepackage{pgfplots}
\usepackage{enumitem}
\usepackage{mathrsfs}
\usepackage{setspace}
\usepackage[normalem]{ulem}
\pgfplotsset{ticks=none}
\setlength\parindent{0pt}
% \usepackage{fancyhdr}
% \pagestyle{fancy}
% \rhead{Gabriel Wallace}
% \lhead{MTH 730}

\newcommand{\N}{\mathbb{N}}
\newcommand{\R}{\mathbb{R}}
\newcommand{\Z}{\mathbb{Z}}
\newcommand{\Q}{\mathbb{Q}}

\newcommand{\inv}{^{-1}}

\newcommand{\Def}{\textbf{Def} }
\newcommand{\Pf}{\textbf{Proof} }
\newcommand{\Qu}{\textbf{Q:} }
\newcommand{\Cor}{\textbf{Corollary} }
\newcommand{\Conj}{\textbf{Conjecture} }
\newcommand{\Note}{\textbf{Note} }

\title{Questions and Conjectures}
\author{Gabriel Wallace}
\date{}

\begin{document}
\maketitle

For a knot, $K$, the following are equivalent:
\begin{enumerate}[label=(\alph*)]
	\item $K$ is tricolorable
	\item $K$ is $S_3$-colorable
	\item $K$ is $S_4$-colorable
\end{enumerate}

\Note For $n \geq 5$, then $S_n$-colorable does not imply $S_{n-1}$-colorable. The proof boils down
to the fact that the only nontrivial normal subgroups of $S_n$ are $A_n$ for $n \geq 5$, and $A_n
\not\cong S_{n-1}$.\\

\Pf $(a) \iff (b)$ and $(c) \implies (b)$ are trivial. Perko proved $(b) \implies (c)$. He used high
level homology. \Qu Is there an easier way?\\

By $G$-colorable, we mean that there is a surjective homomorphism from the knot group, $K$ to $G$.
i.e there exists some $\varphi$ such that
\[\varphi: K \twoheadrightarrow G.\]

\Conj If $T_k$ is $G$-colorable, and \sout{$a,b$ are the labels of any two strands} $a$ is the
overstrand and $b$ is an understrand at any crossing, then $G = \langle a, b \rangle$ and $a, b$ are
conjugate. \\

\Cor $T_k$ is not $\binom{n}{2}$-colorable if $n \geq 4$. \\ 
(Answer to \Qu For what $n$ is a twist knot $\binom{n}{2}$-colorable?)\\

Dr. Rogers provides a short proof of $a, b$ being conjugate. \\

The sketch of the proof is as follows:\\
Without loss of generality, we label $a, b$ at the bottom of the knot. We continue up the twist
portion labeling each strand on the left $a_k$ and the right $b_k$. Note that $b_k = a_{k-1}$. By
observation we see 
\begin{center}
	\begin{tabular}{l c}
		$k$ is odd: & $a_k = (a_{k-1})\inv b_{k-1} a_{k-1}$\\
		$k$ is even: & $a_k = {a_{k-1}} b_{k-1} (a_{k-1})\inv$\\
	\end{tabular}
\end{center}

Define $h = a\inv b$. Then,\\
\begin{center}
\begin{tabular}{l | c | c}
	& $k$ odd & $k$ even\\
	\hline
	& &\\
	Recursive & $a_k = h a_{k-1}$ & $a_k = a_{k-1} h\inv$\\
	& &\\
	Closed & $a_k = h^{\frac{1+k}{2}} a h^{\frac{1-k}{2}}$ & $a_k = h^{\frac{k}{2}} a h^{-\frac{k}{2}}$
	& &\\
\end{tabular}
\end{center}

At the top of the knot, we have two consistency equations, a left and right as follows:
\begin{center}
\begin{tabular}{l | c | c}
	& $n$ odd & $n$ even\\
	\hline
	& &\\
	Left & $a = a_n b a_n \inv$ & $a = a_n\inv b a_n$\\
	& &\\
	Right & $a_n = b b_n b\inv$ & $a_n = b\inv b_n b$
	& &\\
\end{tabular}
\end{center}

\\
It's easy to show that left consistency equations hold if the right consistency equations hold. It's
also easy to show that the equations in the above tables hold by induction.\\

\begin{center}
	\textbf{\Large Questions following the Conjecture}

\end{center}
\begin{enumerate}
	\item Can this generalize to other knots? If so, what conditions need to be applied?
		\begin{itemize}
			\item Prime for sure by paper by Norwood.
		\end{itemize}
	\item Is there a prime knot that needs more than two generators?
\end{enumerate}
\end{center}

\end{document}
